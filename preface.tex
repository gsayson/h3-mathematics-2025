\newenvironment{preface}{
    \vspace*{\stretch{2}}
    {\noindent \bfseries \Huge \prefacename}
    \begin{center}
        \phantomsection \addcontentsline{toc}{chapter}{\prefacename} % enable this if you want to put the preface in the table of contents
        \thispagestyle{plain}
    \end{center}
}
{\vspace*{\stretch{5}}}

\begin{preface}
    When people find out that you're writing a math book, most of them would usually ask
    two things:
    \begin{enumerate}
        \item ``Why are you writing this book?''
        \item ``How is it different from the other books out there?''
    \end{enumerate}
    It is easy to answer the first question: I am simply unsatisfied by the sheer lack of learning material
    for this subject. That is to be expected though, since most students sitting for the H3 Mathematics Syllabus
    are extremely interested in mathematics (or they may just find H2 Mathematics too boring).
    Most students will only be armed with some notes from their teachers, covering only what is necessary.

    In order to cater to aspiring mathematicians, this book provides an in-depth treatment of the relevant concepts.
    In the process, we help the reader build an intuition of concepts from various fields of mathematics, namely
    real analysis, discrete mathematics. We als introduce the reader to some results.

    In Chapters 1 and 2, we outline mathematical language, their rules, and show how they can be used
    in proofs of results. In Chapter 3, we introduce techniques that the reader can use to solve problems.

    Chapter 4, an introduction to limits, covers the basic idea of what a limit is (excluding the
    formal $\epsilon$-$\delta$ and $\epsilon$-$N$ definitions of a limit). It starts off with basic notation,
    before covering in-depth the concept of a limit as the `completion of a graph'. We introduce the idea
    first by approximating an example, before exposing the reader to arithmetic laws of limits and teaching
    the reader how to find limits directly. Other than direct evaluation of limits, the squeeze theorem and
    L'H\^{o}pital's rule are introduced.
    We also discuss continuity, and the intermediate value theorem.
    The chapter concludes with differentiation from first principles,
    before leading to Chapter 5 which discusses improper integrals and their evaluation.

    We next look at reduction formulae in Chapter 6, before moving to the summation of a series using
    the method of differences in Chapter 7. In Chapter 8, the following inequalities are introduced,
    along with an illustration to help the reader explain why the inequality is true:
    \begin{enumerate}
        \item AM-GM (arithmetic mean-geometric mean) inequality
        \item Cauchy-Schwarz inequality
        \item Triangle inequality
    \end{enumerate}
    (Note that inequalities 2 and 3 will only be discussed in an $n$-dimensional Euclidean space $\mathbb{R}^n$.)

    The penultimate chapter of this book, Chapter 9, introduces combinatorics. Namely, the Inclusion-Exclusion Principle
    is discussed. Finally, we conclude with Chapter 10, modular arithmetic.

    This book is intended to serve as a supplementary text, which you can study along with your teachers' notes.
    I wish you the best of luck in your one-year journey of H3 Mathematics.

    \begin{flushright}
        \textit{Gerard Sayson}
    \end{flushright}
\end{preface}