\chapter{Introduction to limits}
The concept of a limit is extremely fundamental to the understanding of calculus, namely differentiation
and integration; their formal definitions are in terms of limits. Here, we will introduce limits for
sequences and functions, their arithmetic operations, the squeeze theorem, L'Hopital's rule, and
apply it to formally defining the derivative.

\section{Limits of sequences and functions}

\subsection{Limits of sequences}
Suppose we have a sequence
\[
a_1,a_2,a_3,a_4,...
\]
If the term $a_n$ converges to a fixed value $L$ as $n$ tends to infinity, then we write
\[
\lim_{n \to \infty} a_n = L
\]
This is read as ``the \textit{limit} of the sequence ${a_n}$ as $n$ tends to infinity is $L$''.

This concept of a limit can be stated with the following definition:
\begin{definition}
    If, for any arbitrarily small positive number $\epsilon$, one can always find a term in a sequence ${a_n}$
    such that $|a_n - L| < \epsilon$ if $n > N(\epsilon)$, and $N(\epsilon)$ is a function of $\epsilon$.
\end{definition}

A more advanced treatment of limits, and how the above definition can prove common limits, can be found in
\cite{Gong_2017}.

Take note that there are sequences which diverge (i.e. they do not converge). Consider the harmonic series,
where the $n$-th term is \[H_n = \sum_{k = 1}^n \frac{1}{k}\] Although it increases slowly, it can be proven
that \[\lim_{n \to \infty} H_n = \infty\] This proof is left as an exercise to the reader.

\subsection{Limits of functions}
Similar to the above interpretation of limits to sequences, suppose that we have a function
$f(x)$. If $f(x)$ approaches $L$ as $x$ tends to $c$ from both sides (i.e., from $-\infty$, and from $\infty$),
then $\lim_{x \to c} f(x) = L$.

\begin{example}
    Find the limit $\lim_{x \to \infty} \frac{1}{x}$.
\end{example}
\begin{solution}
    The limit is \[\lim_{x \to \infty} \frac{1}{x} = 0.\]
\end{solution}

A function $f(x)$ is said to be continuous at $x = c$, if $\lim_{x \to c} = f(c)$.

\section{Operations of limits}
Suppose that $\lim_{x \to c} f(x) = A$ and $\lim_{x \to c} g(x) = B$ for the functions $f(x)$, $g(x)$,
and $A,B$ are constants. Then:
\[
\lim_{x \to c} (f(x) \pm g(x)) = A \pm B,
\]
\[
\lim_{x \to c} (f(x) g(x)) = A B,
\]
\[
\lim_{x \to c} \left(\frac{f(x)}{g(x)}\right)  = \frac{A}{B},\,\,B\ne 0,
\]
\[
\lim_{x \to c} kf(x) = k\lim_{x \to c} f(x) = kA,\,\,k\,\,\text{is a constant.}
\]

\begin{example}
    Find the limit of $k e^{-x}$, as $x$ tends to infinity. 
\end{example}
\begin{solution}
    The limit is \[\lim_{x \to \infty} k e^{-x} = k \lim_{x \to -\infty} e^{-x} = k \times 0 = 0.\]
    One can study the behavior of $e^{-x}$ graphically.
\end{solution}

\section{The squeeze theorem}
The squeeze theorem is extremely useful in determining important limits; in fact, it is key to proving the
derivative of $\sin(x)$ (with respect to $x$). Hence, we begin with a statement of the theorem.

\begin{theorem}[Squeeze theorem]\thlabel{squeeze}
    Suppose that we have three functions $f(x), g(x), h(x)$ such that $g(x)\le f(x)\le h(x)$. Then
    \[
    \lim_{x \to c} g(x) = \lim_{x \to c} h(x) = L \implies \lim_{x \to c} f(x) = L
    \]
    where $c$ is a constant.
\end{theorem}
\begin{proof}
    The proof can be found on Wikipedia.
    % It is beyond the scope of this book.
\end{proof}

Now, we begin with some examples.

\begin{example}
    Find the limit \[\lim_{x \to 0}x^2\sin\left(\frac{1}{x}\right).\]
\end{example}
\begin{solution}
    We cannot apply the law
    \[
    \lim_{x \to c} (f(x) g(x)) = A B
    \]
    because $\lim_{x \to 0} \sin\left(\frac{1}{x}\right)$ does not exist. However, since the range of $\sin(x)$
    is $[-1, 1]$, we can establish the inequality \[-1 \le \sin\left(\frac{1}{x}\right) \le 1.\] Multiplying both
    sides by $x^2$, we obtain \[-x^2 \le x^2 \sin\left(\frac{1}{x}\right) \le x^2.\]
    Evaluating the leftmost and
    rightmost limits by direct substitution,
    \[\lim_{x \to 0} x^2 = \lim_{x \to 0} (-x^2) = 0 \implies
    \lim_{x \to 0} x^2 \sin\left(\frac{1}{x}\right) = 0 \]
    by the squeeze theorem, and we are done.
\end{solution}

\section{Differentiation from first principles}
Differentiation in the `A'-Level and `O'-Level has, in the author's experience, only been taught in terms of
memorizing formulae and identities. For example, one simply assumes that \[\frac{d}{dx}\sin(x) = \cos(x).\]

The proof of this `trivial' identity, is not `trivial'; it has never ever been covered in any
ordinary course. This is because the proof of this identity utilizes \thref{squeeze}.
In fact, most trigonometric identities stem from the aforementioned theorem.

Now, we must start with the very definition of what a
derivative is. Instead of introducing the definition directly without further explanation, let us
briefly derive the definition of the derivative ourselves.

Recall that the gradient of a straight line $y$ passing through two points $(x_1, y_1)$ and $(x_2, y_2)$ is
defined as $m = \frac{y_2 - y_1}{x_2 - x_1}.$, where $x_2 > x_1$ and $x,y \in \mathbb{R}$.
If $y = f(x)$, then $m = \frac{f(x_2) - f(x_1)}{x_2 - x_1}.$

Now, consider the difference between $x_2$ and $x_1$; let this difference be
$\delta = x_2 - x_1$. Then $x_2 = x_1 + \delta$. Replacing all such occurences of $x_2$, one obtains
\[m = \frac{f(x_1 + \delta) - f(x_1)}{\delta}.\]

We have learnt that we can draw a tangent at a point in a graph to find the gradient at that point; this is exactly
the idea we use here to formally define the derivative. As $\delta$ tends to $0$, we will get a better approximation
of the gradient at the point $(x_1, y_1)$. Alas, using limits, we have the following definition:

\begin{definition}
    The derivative of a function $f(x)$, with respect to $x$, is defined as
    \[f'(x) = \lim_{\delta \to 0} \frac{f(x + \delta) - f(x)}{\delta}\]
\end{definition}

When we apply this definition in finding a derivative, the process is dubbed as \textit{differentiating
from first principles}.

\begin{example}
    Find the derivative of $f(x) = 3x^2 + 2x + 1$.
\end{example}
\begin{solution}
    By differentiating from first principles, one obtains
    \begin{equation*}
        \begin{split}
            f'(x) &= \lim_{\delta \to 0} \frac{f(x + \delta) - f(x)}{\delta} \\
            &= \lim_{\delta \to 0} \frac{(3(x + \delta)^2 + 2(x + \delta) + 1) - (3x^2 + 2x + 1)}{\delta} \\
            &= \lim_{\delta \to 0} \frac{3(x + \delta)^2 + 2x + 2\delta + 1 - 3x^2 - 2x - 1}{\delta} \\
            &= \lim_{\delta \to 0} \frac{3(x + \delta)^2 + 2\delta - 3x^2}{\delta} \\
            &= \lim_{\delta \to 0} \frac{3(x^2 + 2x\delta + \delta^2) + 2\delta - 3x^2}{\delta} \\
            &= \lim_{\delta \to 0} \frac{3x^2 + 6x\delta + 3\delta^2 + 2\delta - 3x^2}{\delta} \\
            &= \lim_{\delta \to 0} \frac{6x\delta + 3\delta^2 + 2\delta}{\delta} \\
            &= \lim_{\delta \to 0} 6x + 3\delta + 2 \\
            &= 6x + 3(0) + 2 \\
            &= 6x + 2
        \end{split}
    \end{equation*}
    Indeed, this is what we expect since \[\frac{d}{dx}(3x^2 + 2x + 1) = 6x + 2.\]
    Hence we are done.
\end{solution}

\section{L'Hopital's rule}


\newpage
\section{Exercises}

\begin{exercise}
    Prove that \[\lim_{x \to 0} |f(x)| = 0 \implies \lim_{x \to 0} f(x) = 0\]
    using the squeeze theorem. (Hint: $|-f(x)| = |f(x)|$.)
\end{exercise}

\begin{exercise}
    It is known that the sum of a geometric series up to $n$ terms is given by
    \[S_n = \sum_{k=1}^n ar^{k - 1} = \frac{a(r^n - 1)}{r - 1}\]
    If $|r| < 0$, take the limit of $S_n$ as $n$ approaches $\infty$.
\end{exercise}